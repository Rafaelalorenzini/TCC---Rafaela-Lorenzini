% ||||||||||||||||||||||||||||||||||||||||||||||
% Informações de dados para CAPA e FOLHA DE ROSTO
% ||||||||||||||||||||||||||||||||||||||||||||||
\titulo{Análise da Presença de Trihalometanos na Água para Consumo Humano do Estado do Rio Grande do Sul no período de 2014 a 2020} % Não utilize o ponto final no título
\autor{Rafaela Lorenzini}
\local{Porto Alegre}
\data{2021}
\orientador {Prof.ª Dr.ª Mariliz Gutterres Soares}
\coorientador{MSc. Luciano Barros Zini} % comente esta linha 


\instituicao{%
  UNIVERSIDADE FEDERAL DO RIO GRANDE DO SUL
  \par
  ESCOLA DE ENGENHARIA
  \par
  DEPARTAMENTO DE ENGENHARIA QUÍMICA
    \par
  TRABALHO DE DIPLOMAÇÃO EM ENGENHARIA QUÍMICA
}
\tipotrabalho{Monografia (Graduação)}
% O preambulo deve conter o tipo do trabalho, o objetivo, 
% o nome da instituição e a área de concentração 

\preambulo{Trabalho de Conclusão de Curso apresentado à COMGRAD/ENQ da Universidade Federal do Rio Grande do Sul como parte dos requisitos para a obtenção do título de Bacharel em Engenharia Química.}

%\preambulo{Trabalho apresentado como requisito para a obtenção do título de Mestre, pelo Programa de Pós-Graduação em Engenharia Elétrica da Universidade do Vale do Rio dos Sinos – UNISINOS.}

% ----------------------------------------------
% Configurações de aparência do PDF final
% ----------------------------------------------
% alterando o aspecto da cor azul
